\documentclass[a4paper]{jpconf}
\usepackage{graphicx}
\usepackage{hyperref}

\newcommand{\soft}[1]{\texttt{#1}}
\newcommand{\Sage}{\soft{Sage}{ }}

\begin{document}
\title{Tensor calculus with free software: \\
the SageManifolds project}

\author{Eric Gourgoulhon$^1$, Micha\l{} Bejger$^2$, Marco Mancini$^1$}

\address{$^1$ Laboratoire Univers et Th\'eories, UMR 8102 du 
CNRS, Observatoire de Paris, Universit\'e Paris Diderot,
92190 Meudon, France}

\address{$^2$ Centrum Astronomiczne im. M. Kopernika, ul. Bartycka 18,
00-716 Warsaw, Poland}

\ead{eric.gourgoulhon@obspm.fr}

\begin{abstract}
Abstract
\end{abstract}

\section{Introduction}

Computer algebra for general relativity (GR) has a long history, which started
almost as soon as computer algebra itself in the 1960s. 
The first GR software was \soft{ALAM} (for \emph{Atlas Lisp Algebraic Manipulator})
witten by R.A. d'Inverno in 1969, who used it to compute
the Riemann and Ricci tensors of the Bondi metric. The original calculations took Bondi and his collaborators 6 months to go. The computation with \soft{ALAM} took 4 minutes and yield to the discovery of 6 errors in the original paper \cite{Skea94}. 
Since then, many packages have been developed and the reader is refered to \cite{MacCa02}
for a review of computer algebra
systems for GR prior to 2002 and to \cite{KorolKS13} for a more recent review,
focussing on tensor calculus, as well as to the semi-exhaustive list of
tensor calculus packages at \cite{xact_links}.

%%%%%%%%%%%%%%%%%%%%%%%%%%%%%%

\section{Software for differential geometry}

Software packages for differential geometry and tensor calculus can be 
classified in two categories: 
\begin{enumerate}
\item Applications atop general purpose computer algebra systems; 
notable examples are 
the \soft{xAct} suite \cite{Marti08,xAct} and \soft{Ricci} \cite{Ricci}, both
running atop \soft{Mathematica},
\soft{DifferentialGeometry} \cite{AnderT12,DiffGeom} integrated into \soft{Maple}, and \soft{Atlas 2}
\cite{Atlas2} for both \soft{Mathematica} and \soft{Maple}.
\item Standalone applications; recent examples are \soft{Cadabra} \cite{Peete07,Cadabra} (field theory),
\soft{SnapPy} \cite{SnapPy} (topology and geometry of 3-manifolds) and
\soft{Redberry} (tensors) \cite{BolotP13,Redberry}.
\end{enumerate}
All applications listed in the second category are free (open-source) software. In
the first category, \soft{xAct} and \soft{Ricci} are also free software, but
they run atop a proprietary product, the sources of which are closed (\soft{Mathematica}). 

As far as tensor calculus is concerned, the above packages can be distinguished by 
the type of computation that they perform: abstract index manipulations 
(\soft{xAct/xTensor}, \soft{Ricci}, \soft{Cadabra}, \soft{Redberry})
or component calculus (\soft{xAct/xCoba}, \soft{DifferentialGeometry}, \soft{Atlas 2})
In the first category, tensor operations such as contraction or covariant differentiation 
are performed by manipulating the indices themselves rather than the components 
to which they correspond. In the second category, vector bases are explicitely 
introduced on manifolds and tensor operations are carried out on the components 
in a given basis.


%%%%%%%%%%%%%%%%%%%%%%%%%%%%%%

\section{An overview of Sage}

\Sage \cite{sage} is a free open-source mathematics software system, which is
based on the Python programming language. It makes use of 90 open-sources packages, 
among which \soft{Maxima} and \soft{Pynac} (symbolic calculations),
\soft{GAP} (group theory), 
\soft{PARI/GP} (number theory), \soft{Singular} (polynomial computations), 
and \soft{matplotlib} (high quality 2D figures). 
\Sage provides a uniform Python interface to all these packages; however, 
\Sage is much more than a mere interface: it contains a large and increasing part of 
original code (more than 750,000 lines of Python and Cython, involving 5344 classes). 
\Sage has been created in 2005 by W. Stein \cite{SteinJ05} and since
then its development has been sustained by more than a hundred researchers
(mostly mathematicians). Good introductory textbooks about \Sage are
\cite{JoyneS14,Zimme13,Bard15}. 
 
 
\ack
This work has benifited from enlightening discussions with Volker Braun,
Vincent Delecroix, Simon King,  Jos\'e M. Mart\'\i n-Garc\'\i a, 
S\'ebastien Labb\'e,
Marc Mezzarobba, Thierry Monteil, Travis Scrimshaw and Nicolas Thi\'ery. 
We also thank St\'ephane M\'en\'e for his technical help. 
 

%%%%%%%%%%%%%%%%%%%%%%%%%%%%%%%%%%%%%%%%%%%%%%%%%%%%%%%%%

\section*{References}
\begin{thebibliography}{10}
\bibitem{Skea94}
Skea J E F 1994 Applications of SHEEP {\it Lecture notes available at}
\url{
http://www.computeralgebra.nl/systemsoverview/special/tensoranalysis/sheep/}
\bibitem{MacCa02}
MacCallum M A H 2002 {\it Int. J. Mod. Phys. A} {\bf 17}, 2707 
\bibitem{KorolKS13}
Korol'kova A V, Kulyabov D S and Sevast'yanov L A 2013 {\it Prog. Comput. Soft.} 
{\bf 39}, 135
\bibitem{xact_links} 
\url{http://www.xact.es/links.html}
\bibitem{Marti08}
Martin-Garcia J M 2008 {\it Comput. Phys. Commun.} {\bf 179}, 597
\bibitem{xAct}
\url{http://www.xact.es}
\bibitem{Ricci}
\url{http://www.math.washington.edu/~lee/Ricci/}
\bibitem{AnderT12}
Anderson I M and Torre C G 2012 {\it J. Math. Phys.} {\bf 53}, 013511
\bibitem{DiffGeom}
\url{http://digitalcommons.usu.edu/dg/}
\bibitem{Atlas2}
\url{http://digi-area.com/Maple/atlas/}
\bibitem{Peete07}
Peeters K 2007 {\it Comput. Phys. Commun.} {\bf 15}, 550
\bibitem{Cadabra}
\url{http://cadabra.phi-sci.com/}
\bibitem{SnapPy}
Culler M, Dunfield N M and Weeks J R, SnapPy, a computer program for studying the geometry and topology of 3-manifolds, \url{http://snappy.computop.org}
\bibitem{BolotP13}
Bolotin D A and Poslavsky S V 2013 Introduction to Redberry: the computer algebra system designed for tensor manipulation {\it Preprint} arXiv:1302.1219
\bibitem{Redberry}
\url{http://redberry.cc/}
\bibitem{sage}
\url{http://sagemath.org/}
\bibitem{SteinJ05}
Stein W and Joyner D 2005 {\it Commun. Comput. Algebra} {\bf 39}, 61
\bibitem{JoyneS14}
Joyner D and Stein W 2014 {\it Sage Tutorial} (CreateSpace)
\bibitem{Zimme13}
Zimmermann P et al. 2013 {\it Calcul math\'ematique avec Sage} (CreateSpace); 
freely downloadable from \url{http://sagebook.gforge.inria.fr/}
\bibitem{Bard15}
Bard G V 2015 {\it Sage for Undergraduates} (Americ. Math. Soc.) in press;
preprint freely downloadable from \url{http://www.gregorybard.com/})
\end{thebibliography}

\end{document}
