\chapter{Conclusion and perspectives} \label{s:con}

We have presented some aspects of symbolic tensor calculus as implemented
in \Sage{}. The implementation is independent of the symbolic engine (i.e. the
tool used to performed symbolic calculus on coordinate representations of
scalar fields), the latter being involved only in the last stage of the diagram
shown in Fig.~\ref{f:vec:storage_tensor}.

The implementation has been performed via the \soft{SageManifolds}
project, the home page of which we refer for details and material complementary
to what has been shown here (in particular many more examples):
\begin{center}
\url{http://sagemanifolds.obspm.fr/}
\end{center}
This project resulted in $\sim 75,000$ lines of Python code (including comments and doctests), which have been submitted to \Sage{} community as a sequence of
33 tickets\footnote{Cf. the meta-ticket \url{https://trac.sagemath.org/ticket/18528}.},
at the time of this writing (March 2018). The
first ticket was accepted in March 2015 and the 33th one in March 2018.
These tickets have been written and reviewed by a dozen of
contributors\footnote{Cf. the list at \url{http://sagemanifolds.obspm.fr/authors.html}.}.
As a result, all code is fully included in \soft{SageMath~8.2} and does not require
any separate installation. The following features have been already implemented:
\begin{itemize}
\item differentiable manifolds: tangent spaces, vector frames, tensor fields, curves, pullback and pushforward operators;
\item standard tensor calculus (tensor product, contraction, symmetrization, etc.), even on non-parallelizable manifolds;
\item all monoterm tensor symmetries taken into account;
\item Lie derivatives of tensor fields;
\item differential forms: exterior and interior products, exterior derivative,
Hodge duality;
\item multivector fields: exterior and interior products, Schouten-Nijenhuis bracket;
\item affine connections (curvature, torsion);
\item pseudo-Riemannian metrics;
\item computation of geodesics (numerical integration via \Sage{}/\soft{GSL});
\item some plotting capabilities (charts, points, curves, vector fields);
\item parallelization (on tensor components) of CPU demanding computations,
via the Python library \code{multiprocessing};
\item the possibility to use \soft{SymPy} as the symbolic engine, instead of
\Sage{}'s default, which is \soft{Pynac} (with \soft{Maxima} for simplifications).
\end{itemize}
The \soft{SageManifolds} project is still ongoing and future prospects include
\begin{itemize}
\item more symbolic engines (\soft{Giac}, \soft{FriCAS}, ...);
\item extrinsic geometry of pseudo-Riemannian submanifolds;
\item integrals on submanifolds;
\item more graphical outputs;
\item more functionalities: symplectic forms, fibre bundles,
spinors, variational calculus, etc.;
\item connection with numerical relativity: using \Sage{} to explore
numerically-generated spacetimes, by using \emph{numerical} engines, instead
of \emph{symbolic} ones, in the last stage of the diagram
shown in Fig.~\ref{f:vec:storage_tensor}.
\end{itemize}
In the spirit of open-source software, anybody interested is welcome
to join the project. Please visit
\begin{center}
\url{http://sagemanifolds.obspm.fr/contact.html}
\end{center}

